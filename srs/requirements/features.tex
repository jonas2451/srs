\begin{comment}
    \subsection{$<$feature name$>$}
        \subsubsection{Introduction/Purpose of feature}
        \subsubsection{Stimulus/Response sequence}
        \begin{description}
            \item[Stimulus:]
            \item[Response:]
        \end{description}  

    \subsection{$<$Login$>$}
    \subsubsection{Introduction/Purpose of feature}
    To gain access the webapplication (excluding the forum) the user shall be logged in. 
    \subsubsection{Stimulus/Response sequence}
    \begin{description}
        \item[Stimulus: The user wants to access the application]
        \item[Response: The system offers posibility to login]
        \item[Stimulus: The user enters his password and username] 
        \item[Response: The system grants the user access if his login data is correct]
    \end{description} 
    On the student website are different levels of access. 
    In order to differenciate between users and identify which areas are accessible, users have to login. 
    The forum is an exception and does not require a login, to READ threads. 
\end{comment}


    \begin{tabular}{ |p{2cm}||p{11cm}| }
        \hline
        \multicolumn{2}{|c|}{$$Name$$} \\ \hline
        \textbf{ID} & \textbf{FT01} \\ \hline
        Use Case & \underline{UC001} \\ \hline
        Purpose & test \\ \hline
        Description &
        \begin{enumerate}
            \item Stimulus: 
            \item Response: 
        \end{enumerate}
        \\ \hline 
        Rationale & test \\ \hline
    \end{tabular}


    \subsection{$<$Register User$>$}
        \subsubsection{Introduction/Purpose of feature}
        In order to be administrated a user first shall be registered. Also a user must be registered to login. 
        \subsubsection{Stimulus/Response sequence}
        \begin{description}
            \item[Stimulus: An administrator ]
            \item[Response:]
        \end{description}